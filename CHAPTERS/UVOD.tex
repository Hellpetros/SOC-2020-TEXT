\chapter*{Úvod}
\addcontentsline{toc}{chapter}{Úvod}

Zahradničení je dnes naprosto běžnou zájmovou činností. Mnoho lidí majících takovou zálibu je ovšem velmi časově vytížených. Kromě práce se musí starat mnohdy i o~rodinu a~na péči o~rostliny jim často jednoduše nezbývá čas. Jedním z~těchto lidí je i můj táta, který mě inspiroval k~vytvoření PROTOPlantu - systému pro snadnou a~levnou automatizaci skleníku.

Cílem této práce je vytvořit univerzální a~dostupný systém pro automatizaci skleníku, který by usnadnil péči o~rostliny časově vytíženým lidem. 

Systémy pro takovouto automatizaci dnes existují, jsou ovšem určené primárně pro velkozemědělství, nikoli pro člověka, který ve skleníku pěstuje několik druhů zeleniny pro sebe proto, aby ji nemusel kupovat v~obchodě, nebo který vlastní menší skleník s~okrasnými rostlinami. 

Samozřejmě, na internetu existuje spousta návodů, jak si nějaký takový „systém“ vyrobit za pomoci Arduina, nepájivého pole a~\uv{pár} drátků.
Takové řešení se mi ovšem nezdá příliš univerzální a~pracující lidé nemají mnohdy čas si takto hrát.
Zároveň pro sestavení něčeho takovového potřebují mít určité znalosti v~elektrotechnice.

Kromě toho jsem chtěl, aby bylo možno systém v~budoucnu připojit k~internetu a~sledovat jej tak například z~dovolené, případně měnit nastavení.

Při vytváření práce jsem si dal za cíl, aby byl systém:
\begin{itemize}
    \item kompletně open-source
    \item levný
    \item modulární
    \item snadný na ovládání
    \item univerzální
\end{itemize}

Dalším z~cílů tohoto projektu je úspora energií (elektřina, voda), které lze díky automatizaci dosáhnout.

V~průběhu práce jsem systém nazval \B{PROTOPlant}.

\newpage
