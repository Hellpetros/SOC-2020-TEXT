\chapter*{Úvod}
\addcontentsline{toc}{chapter}{Úvod}
Pěstování skleníkových rostlin je dnes naprosto běžnou zájmovou činností. 
Ať již člověk pěstuje zeleninu, orchideje, nebo technické plodiny, vždy dojde k~závěru, že na ně mnohdy nemá čas.
Když přijde domů z~práce, musí se postarat o~rodinu, připadně dodělat jiné činnosti a~na rostliny již jednoduše čas nezbývá.
Jedním z~těchto lidí je i můj táta.
Velmi rád pěstuje orchideje, které má ve skleníku.
Ovšem postupem času na ně má z~pracovních důvodů stále méně a~méně času.
Proto mne napadlo, že lidí, kteří na tom jsou tak, jako on, je jistě mnoho.
To mne inspirovalo k~vytvoření PROTOPlantu -- systému pro levnou a~snadnou automatizaci skleníku dostupného každému. 

Cílem této práce je vytvořit univerzální a~dostupný systém pro automatizaci skleníku, který by usnadnil péči o~rostliny časově vytíženým lidem. 

Systémy pro takovouto automatizaci dnes existují, jsou ovšem určeny primárně pro velkozemědělství, nikoli pro člověka, který ve skleníku pěstuje několik druhů zeleniny pro sebe, aby ji nemusel kupovat v~obchodě nebo který vlastní menší skleník s~okrasnými rostlinami. 

Samozřejmě, na internetu existuje spousta návodů, jak si nějaký takový „systém“ vyrobit za pomoci Arduina, nepájivého pole a~\uv{pár} drátků.
Takové řešení se mi ovšem nezdá příliš univerzální a~pracující lidé nemají mnohdy čas si takto hrát.
Zároveň pro sestavení něčeho takového potřebují mít určité znalosti v~elektrotechnice a~programování.

Kromě toho jsem chtěl, aby bylo možno systém v~budoucnu připojit k~internetu a~sledovat jej tak například z~dovolené.
\newline

Při vytváření práce jsem si dal za cíl, aby byl systém:
\begin{itemize}
    \item kompletně open-source
    \item levný
    \item modulární
    \item snadný na ovládání
    \item univerzální
\end{itemize}

Dalším z~cílů tohoto projektu je úspora energií (elektřina, voda), které lze díky automatizaci dosáhnout.

\newpage
