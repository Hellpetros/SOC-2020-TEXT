\chapter{Dlouhodobé testování}
Již přibližně rok je PROTOPlant spuštěn v našem domácím skleníku s orchidejemi.
Během této doby jsem na této jednotce testoval stabilitu a funkčnost celého systému. 
Zároveň jsem zde pravidelně aktualizoval software a s pomocí této testovací jednotky jsem zjišťoval chyby v softwaru a mohl je průběžně opravovat.
Ke dni \It{10. 2. 2020} je aktuálně instalovanou verzí PROTOPlant v4.9.

\section{Testování napájení solární energií}
\label{sec:SolarPower}
Jednotka PROTOPlantu v tomto skleníku je napájena čistě solární energií získávanou ze dvou solárních panelů instalovaných na střeše přilehlého zahradního domku.
Přebytečná energie je ukládána do dvou olověných akumulátorů s napětím 12~VDC.
Z nich je systém napájen v noci, nebo v počasí se špatnou viditelností.

Díky tomu je tato jednotka \B{naprosto nezávislá} na dodávce energie z elektrické sítě.
Jednotka tedy běží i v případě výpadku energie a její náklady na provoz jsou tedy \B{nulové}.

Zásobování vodou řeší sběr dešťové vody ze střechy nedalekého rodinného domu do nádrže instalované pod zemí vedle skleníku samotného.

\section{Průběh testů}
Za dobu fungování proběhlo několik testů v kratších i delších obdobích. \newline

\noindent\B{Co se týče kratších testů}, jejich délka byla v řádech dní či týdnů. 
Tyto testy probíhaly většinou s cílem testování stability softwaru.
V jejich průběhu mnohdy probíhaly různé opravy softwaru, případně nápravy případných hardwarových nedostatků. \newline

\noindent\B{Dlouhodobé testy} probíhaly většinou s cílem otestovat přesnost senzorů, stabilitu systému či jeho energetickou náročnost.
Předpokladem těchto testů bylo použití \B{stabilní verze softwaru}. 

\section{Výsledky testů}
Provedené testy ukázaly, že PROTOPlant je schopen fungovat naprosto samostatně bez potřeby zásahu uživatele i po dobu několika měsíců (bereme-li v potaz pouze dlouhodobé testy viz výše).

\newpage