\chapter{Úspora energií dosažená s~pomocí PROTOPlantu}
Díky možnosti napájet PROTOPlant \B{kompletně s~pomocí solární energie} lze ušetřit elektrickou energii. 
V~noci jej lze napájet z~baterií, do kterých se přebytková energie ze solárních panelů ukládá.
Tento způsob napájení již dlouhou dobu testuji na našem domácím skleníku. 
O~výsledcích se dočtete v~kapitole \ref{sec:SolarPower}.

O solárních panelech se obecně ví, že jejich počáteční náklady jsou poměrně vysoké, ovšem existují způsoby, jak ušetřit i v tomto ohledu.
Existuje mnoho lidí, kteří prodávají solární panely z druhé ruky například ze zrušených solárních farem.
Tyto panely sice mají nižší účinnost, ovšem je otestované, že PROTOPlant dokáží napájet bez problému.
Společně s těmito panely se dají zakoupit i použité akumulátory (povětšinou NiCd FERAK).
Z vlastní zkušenosti mohu potvrdit, že tyto akumulátory jsou ideální pro skladování energie získané pomocí solárních panelů.

Dále lze díky němu šetřit vodou, protože se vždy zavlažuje pouze ve chvíli, kdy je to potřeba.

\newpage