\chapter{Dostupnost a distribuce PROTOPlantu}
Jak popisuji v úvodu, jedním z cílů, který jsem si dal na začátku práce na PROTOPlantu bylo šíření pod licencí open-source.
Toto jsem dodržel. 
Celý software PROTOPlantu je šířen pod licencí MIT, ostatní části (Hardware, atd.) včetně textu této práce je poté pod CC BY-NC-SA 4.0.
Co se týče hardwaru, převzatý hardware (senzory, procesor a další moduly) je z licence vyňat.
Člověk, který tedy elektrotechnice a programování rozumí si poté může PROTOPlant bez problému sestavit sám v pohodlí domova.

Ovšem stále je obrovské množství lidí, kteří na sestavení PROTOPlantu nemusí mít dostatečné znalosti, nebo nemají čas si jej sestavovat.
Z toho důvodu plánuji zahájit výrobu a distribuci mého systému jakožto hotových komponent, které stačí nainstalovat a zapojit.
Ke dni \It{20. 2. 2020} jsem ve fázi, kdy připravuji výrobní podklady jednotlivých komponent, a provádím kroky vedoucí k založení podniku.
Již v minulém roce jsem provedl průzkum, který ukázal, že o PROTOPlant by skutečně byl zájem.

\newpage