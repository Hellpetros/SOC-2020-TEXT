\chapter{Software základní desky}
Tato kapitola se zaměřuje na software základní desky ProtoPlantu a detailně popisuje jeho funkci.
Na software ostatních modulů se zaměřuje následující kapitola \ref{chap:moduleSoftware}.

\paragraph{Blokové schéma funkce softwaru základní desky}
Schéma funkce softwaru základní desky je shrnuto blokovým diagramem XXX

\section{Sdílené knihovny}
Z důvodu usnadnění programování základní desky i ostatních rozšiřujících modulů jsem vytvořil několik sdílených knihoven. 
V nich je zahrnuto:
\begin{itemize}
    \item konfigurace systému
    \item nastavení jednotlivých pinů dle standartního rozložení, vč. možnosti nastavení vlastního
    \item práce s displayem
    \item práce s tlačítky
    \item řízení H-můstků
    \item ovládání senzorů
\end{itemize}
Díky těmto knihovnám je většina zdrojového kódu uložena v nich. 
Koncový uživatel, který se rozhodne software modifikovat, poté pouze v hlavním programu definuje, které moduly spustit a do konfiguračního souboru zapíše nastavení daných modulů.

\paragraph{Konfigurace softwaru}

\section{Datové sběrnice}
ProtoPlant primárně využívá dvě datové sběrnice:
\begin{itemize}
    \item I\superscript{2}C
    \item OneWire
\end{itemize}

\paragraph{Sběrnice I\superscript{2}C}

\paragraph{Sběrnice OneWire}

\section{Komunikace mezi jednotlivými moduly}

\section{Bezdrátová komunikace}


\chapter{Software dalších modulů}
\label{chap:moduleSoftware}

\newpage