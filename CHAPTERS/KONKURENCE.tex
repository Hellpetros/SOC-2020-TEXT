\chapter{Konkurence}
Potenciální konkurenci PROTOPlantu jsem rozdělil do dvou kategorií:
\begin{itemize}
    \item průmyslová řešení
    \item podomácku vytvořená „kutilská“ řešení
\end{itemize}

\section{Průmyslová řešení}
Systémy spadající do této kategorie vyznačují se především tím, že jsou:
\begin{itemize}
    \item určena primárně pro velkozemědělství
    \item drahá
    \item vyráběna na zakázku (nejsou tedy příliš univerzální)
    \item nedostupná pro běžné zákazníky
\end{itemize}

Tato řešení jsou většinou řízena pomocí PLC, tedy průmyslových kontrolérů, která jsou povětšinou velmi drahá a~náročná na programování.
Dodávají je společnosti, které se touto problematikou přímo zabývají a~systémy dodávají na zakázku velkým podnikům.

\section{„Kutilská“, neboli amatérská řešení}
Jsou vytvořena lidmi, kteří elektrotechnice rozumí a~mají čas na to si s~elektronikou takto hrát.
Taková řešení většinou jsou:
\begin{itemize}
    \item neuniverzální, lidé si je vytváří přímo pro sebe
    \item časově nákladná na výrobu
    \item nevhodná pro osoby, které nemají zkušenosti v~elektrotechnice a~programování
\end{itemize}

Lze je sice nazvat levnými, ovšem na úkor jejich funkčnosti.

Pro konkrétní porovnání mohu uvést dva příklady:
\subsection{Zavlažovací systém skleníku \cite{ARDUGREENHOUSE}}
Tento systém založený na Arduinu je dle proklamací autora schopen:
\begin{itemize}
    \item otevírat okna
    \item zavlažovat rostliny
    \item sledovat hladinu vody v~nádrži
\end{itemize}

Co se týče softwaru tohoto řešení, je realizován naprosto amatérsky.
Autor v~něm používá naprosto běžně tzv. blokující funkce, což může v~případě, že nastane nějaká mezní situace způsobit i poškození rostlin, či systému samotného.

Dle autorova tvrzení systém funguje, ovšem byl testován pouze na malém stolním pařeništi.
Samotné provedení považuji za velmi neprofesionální.

\subsection{RaspberryPI greenhouse \cite{RPIGREENHOUSE}}
Projekt založený na RaspberryPI. 
Dokáže měřit teplotu, vlhkost,s automaticky zavlažovat rostliny a~otevírat okna.
Toho je ovšem dosaženo naprosto amatérským provedením.

\section{Srovnání s~PROTOPlantem}
Pro srovnání PROTOPlantu a~těchto dvou kategorií jsem sestavil tabulku \ref{tab:COMPARATION}.

\begin{table}[h]
    \centering
    \resizebox{\textwidth}{!}{%
    \begin{tabular}{@{}l|lll@{}}
        & \textbf{Průmyslová řešení} & \textbf{„Kutilská“ řešení} & \textbf{PROTOPlant}                                                                                   \\ \midrule
    \textbf{Dodání}      & Výroba na zakázku          & „Vyrob si sám“             & \begin{tabular}[c]{@{}l@{}}Možnost sestavení přímo doma, \\ nebo dodání \B{hotového} systému\end{tabular} \\
    \textbf{Cena}        & Drahá ($>$~10~000~Kč)      & Levná ($<$~10~000~Kč)      & Kompromis cena -- výkon (již od 2~500~Kč)                                                                               \\
    \textbf{Ovládání}    & Komplexní                  & Jednoduché (většinou)      & Jednoduché                                                                                            \\
    \textbf{Konektivita} & Většinou ethernet          & Často Wi-Fi                & \begin{tabular}[c]{@{}l@{}}Wi-Fi, Bluetooth, \\ možnost přidání podpory Ethernetu\end{tabular}        \\
    \textbf{Řízení}      & PLC                        & Většinou Arduino           & ESP32                                                                                                 \\ \bottomrule
    \textbf{Modularita}  & Ano                        & Ne                         & Ano                                                                                                   \\
    \textbf{Univerzálnost}& Ano                       & Ne                         & Ano                                                                                                   \\ 
    \textbf{Open-source} & Ne                         & Většinou ano               & Ano                                                                                                   \\ \bottomrule    
    \end{tabular}%
    }
    \caption{Tabulka srovnání PROTOPlantu a~jiných řešení.}
    \label{tab:COMPARATION}
\end{table}

\noindent\B{Dodání} -- udává způsoby, jak si lidé mohou dané řešení opatřit.
„Kutilská“ řešení nejsou k~dostání jako již hotový produkt, zatímco PROTO\-Plant \B{ano}. \newline
Průmyslová řešení jsou většinou již k~dostání jako hotové produkty. \newline

\noindent\B{Cena} -- srovnání cen jednotlivých řešení.
Průmyslová řešení jsou komplexní, ovšem velmi drahá.
„Kutilská“ jsou pravým opakem, ovšem nemají tolik funkcí, jako PROTOPlant, nebo řešení průmyslová.
PROTOPlant má vše, co běžný majitel skleníku potřebuje, za rozumnou cenu. \newline

\noindent\B{Ovládání} -- složitost/jednoduchost ovládání daného systému.
PROTOPlant je navržen tak, aby jeho ovládání zvládnul každý.
Naopak průmyslová řešení mnohdy vyžadují proškolení personálu.
Řešení „kutilská“ dokáže mnohdy ovládat pouze člověk, který si dané řešení navrhnul. \newline

\noindent\B{Konektivita} -- zda systém disponuje možností propojení s~okolním světem.
Průmyslovým standardem je dnes ethernet, tudíž většina průmyslových řešení je vybavena právě jím.
PROTOPlant nativně podporuje Wi-Fi a~Bluetooth a~je možno jej rozšířit o~podporu Ethernetu.
„Kutilská“ řešení jsou většinou vybavena Wi-Fi. \newline

\noindent\B{Řízení} -- řídící jednotka daného řešení.
PLC jsou velmi drahá a~jejich programování vyžaduje školení.
Arduino je levné, ovšem jeho funkce jsou velmi omezené.
ESP32 je výkonný procesor za rozumnou cenu, který nativně podporuje Wi-Fi a~Bluetooth. \newline

\section{Sumarizace}
Nabídka řešení v~tomto oboru je velmi chudá.
Na jedné straně stojí průmyslová řešení, která jsou drahá a~dodávají se primárně do skleníků velkých zemědělských firem.
Na opačném břehu jsou řešení amatérská. 
Ta jsou ovšem buďto naprosto nepoužitelná v~běžně velkém skleníku (primárně je uživatelé vyrábí pro malá pařeniště), nebo neuniverzální.

\newpage