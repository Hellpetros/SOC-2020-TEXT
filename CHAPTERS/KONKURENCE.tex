\chapter{Konkurence}
Potenciální konkurenci PROTOPlantu jsem rozdělil do dvou kategorií:
\begin{itemize}
    \item průmyslová řešení
    \item podomácku vytvořená „kutilská“ řešení
\end{itemize}

\section{Průmyslová řešení}
Systémy spadající do této kategorie vyznačují se především tím, že jsou:
\begin{itemize}
    \item určena primárně pro velkozemědělství
    \item drahá
    \item vyráběna na zakázku (nejsou tedy příliš univerzální)
    \item nedostupná pro běžné zákazníky
\end{itemize}

Tato řešení jsou většinou řízena pomocí PLC, tedy průmyslových kontrolérů, která jsou povětšinou velmi drahá a náročná na programování.
Dodávají je společnosti, které se touto problematikou přímo zabývají a systémy dodávají na zakázku velkým podnikům.

\section{„Kutilská“ řešení}
Jsou vytvořena lidmi, kteří elektrotechnice rozumí a mají čas na to si s~elektronikou takto hrát.
Taková řešení většinou jsou:
\begin{itemize}
    \item neuniverzální, lidé si je vytváří přímo pro sebe
    \item časově nákladná na výrobu
    \item nevhodná pro osoby, které nemají zkušenosti v~elektrotechnice 
\end{itemize}

\section{Srovnání s~PROTOPlantem}
Pro srovnání PROTOPlantu a těchto dvou kategorií jsem sestavil tabulku \ref{tab:COMPARATION}.

\begin{table}[h]
    \centering
    \resizebox{\textwidth}{!}{%
    \begin{tabular}{@{}l|lll@{}}
        & \textbf{Průmyslová řešení} & \textbf{„Kutilská“ řešení} & \textbf{PROTOPlant}                                                                                   \\ \midrule
    \textbf{Dodání}      & Výroba na zakázku          & „Vyrob si sám“             & \begin{tabular}[c]{@{}l@{}}Možnost sestavení přímo doma, \\ nebo dodání hotového systému\end{tabular} \\
    \textbf{Cena}        & Drahé                      & Levné                      & Kompromis cena - výkon                                                                               \\
    \textbf{Ovládání}    & Komplexní                  & Jednoduché (většinou)      & Jednoduché                                                                                            \\
    \textbf{Konektivita} & Většinou ethernet          & Často Wi-Fi                & \begin{tabular}[c]{@{}l@{}}Wi-Fi, Bluetooth, \\ možnost přidání podpory Ethernetu\end{tabular}        \\
    \textbf{Řízení}      & PLC                        & Většinou Arduino           & ESP32                                                                                                 \\ \bottomrule    
    \end{tabular}%
    }
    \caption{Tabulka srovnání PROTOPlantu a jiných řešení.}
    \label{tab:COMPARATION}
\end{table}

\noindent\B{Dodání} -- udává způsoby, jak si lidé mohou dané řešení opatřit.

\noindent\B{Cena} -- srovnání cen jednotlivých řešení.

\noindent\B{Ovládání} -- složitost/jednoduchost ovládání daného systému.

\noindent\B{Konektivita} -- zda systém disponuje možností propojení s~okolním světem.

\noindent\B{Řízení} -- řídící jednotka daného řešení.

\newpage