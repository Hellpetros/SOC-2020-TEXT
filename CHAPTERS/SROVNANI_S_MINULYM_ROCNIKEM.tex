\chapter{Změny oproti minulému roku}
Na konci minulého roku byl PROTOPlant schopen automaticky regulovat teplotu otevíráním oken, případně spínáním topného tělesa, spínat čerpadla zavlažování, a sbírat data o~vlhkosti a teplotě vzduchu. V~tomto roce jsem se zaměřil primárně na přidávání funkcí dalších, sekundárně pak na vylepšování těch stávajících.
Největšími změnami jsou: 
\begin{itemize}
    \item nádstavba softwaru pro implementaci vzdáleného ovládání a sledování
	\item kompletní přepsání softwaru do systému knihoven
	\item výroba a použití vlastních tištěných spojů
	\item implementace frameworku pro měření vlhkosti půdy na jednotlivých místech
	\item implementace podpory senzorů BME280 od Bosch sensortec
\end{itemize}

Dále jsem s~pomocí testovací jednotky instalované ve zkušebním skleníku provedl dlouhodobý test, zaměřený na testování konzistence hodnot naměřených senzory a na spolehlivost PROTOPlantu jako celku. Výsledky byly uspokojující, až na několik poznatků, které jsem využil pro další vylepšování tohoto systému. Mezi tyto poznatky patří:
\begin{itemize}
    \item fluktuace dat čtených ze senzorů DHT11 - v~průběhu testu jsem tyto senzory nahradil přesnějšími DHT22
    \item problém s~operační pamětí - vyřešen implementací automatického restartu pro vyčištění mezipaměti po týdnu běhu
\end{itemize}

Další, spíše formální změnou je úprava licence. Nově je celý PROTOPlant kompletně open-source, včetně HW specifikací.

V~průběhu tohoto roku jsem navrhnul několik DPS pro PROTOPlant. 
Jejich schémata vč. rozložení jsou dostupná na mém GitHubu v~repozizáři PROTOPlant-HW. 
\newpage