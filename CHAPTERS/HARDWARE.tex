\chapter{Hardware}
CHPTR.

\section{Tištěné spoje}
Všechny prototypy základních desek ProtoPlantu byly založeny na univerzálních tištěných spojích. Vzhledem k~tomu, že jsem po stránce vzhledu i funkčnosti nebyl s~takovýmto provedením spokojen, rozhodl jsem se nechat vyrobit vlastní tištěné spoje pro základní desku i senzorové moduly.
Díky tomuto jsem se naučil návrhu tištěných spojů a tvorbě výrobních podkladů v~programu Autodesk EAGLE.

\subsection{PPMB32 - Základní deska}
Základní deska je rozdělena do několika částí. 
Vzhledem k~tomu, že umím pájet velmi dobře, rozhodl jsem se pro ruční osazení všech součástek, které byly doposud osazeny pouze na různých modulech připojených k~základní desce, včetně procesoru ESP32-WROOM32D.
Z~důvodu přehlednosti jsem desku rozdělil do několika částí:

\begin{itemize}
    \item Control (ESP32-WROOM32D a programátor)
    \item H-power (napájecí obvod a H-můstky)
    \item Sin (SensorIN - piny pro připojení senzorů)
    \item Pout (PowerOUT - výstup pro napájení dalších periferií)
    \item PanCon (PanelConnect - piny pro připojení tlačítek a displaye na ovládacím panelu)
\end{itemize} 

Samotná základní deska má dvě verze. Jejich rozdíly jsou vysvětleny níže.

\paragraph{PPMB32-E}
Vzhledem k~tomu, že je ProtoPlant veřejně dostupný, nebyl jsem si jist, zda by kompletní osazení takto velké desky zvládl i laik. 
Napadlo mě proto vytvořit i druhou desku, na které by byly osazeny dutinkové lišty pro vsazení vývojové ESP32 DevKitC. 
Odpadla by tedy nutnost kompletně osazovat sekci Control. 
Tuto verzi jsem nazval PPMB32-E (označení E od anglického slova Easy - jednoduchý).

\paragraph{PPMB32-F}
Kompletní, samostatná deska. 
Je přímo osazena procesorem ESP32-WROOM32D i programátorem. 
Vzhledem k~nepoužití DevKitu C je má deska nižší profil, tudíž je možné ji umístit i do nižších prostor. 
Integrovaný programátor lze s~pomocí jumperů odpojit a přes programovací piny připojit externí.

\paragraph{Sekce Control}
Jak již bylo zmíněno, tato část desky zahrnuje modul procesoru ESP32-WROOM32D a programovací obvod. 
Ten se skládá z~převodníku USB-UART CP2102N, tranzistorů SS8050-G (sloužících pro reset procesoru), indikačních LED diod a mikro USB konektoru.

\subsection{PPSB - Deska se senzory teploty a vlhkosti}

\section{Hardwarové verze ProtoPlantu a jejich odlišnosti}

\section{Krabice pro řídící elektroniku a jejich interiér}

\subsection{Instalace elektroniky do krabic - tzv. StoryMount}

\subsection{Těsnění}

\subsection{Ochrana elektroniky před vlhkostí}

\subsection{Ochrana před přehřátím}

\newpage