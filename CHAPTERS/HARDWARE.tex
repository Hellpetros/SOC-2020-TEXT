\chapter{Hardware}
CHPTR.

\section{Tištěné spoje}
Všechny prototypy základních desek ProtoPlantu byly založeny na univerzálních tištěných spojích. Vzhledem k~tomu, že jsem po stránce vzhledu i funkčnosti nebyl s~takovýmto provedením spokojen, rozhodl jsem se nechat vyrobit vlastní tištěné spoje pro základní desku i senzorové moduly.
Díky tomuto jsem se naučil návrhu tištěných spojů a tvorbě výrobních podkladů v~programu Autodesk EAGLE.

\subsection{PPMB32 - Základní deska}
Základní deska je rozdělena do několika částí. 
Vzhledem k~tomu, že umím pájet velmi dobře, rozhodl jsem se pro ruční osazení všech součástek, které byly doposud osazeny pouze na různých modulech připojených k~základní desce, včetně procesoru ESP32-WROOM32D.
Z~důvodu přehlednosti jsem desku rozdělil do několika částí:

\begin{itemize}
    \item Control (ESP32-WROOM32D a programátor)
    \item H-power (napájecí obvod a H-můstky)
    \item Sin (SensorIN - piny pro připojení senzorů)
    \item Pout (PowerOUT - výstup pro napájení dalších periferií)
    \item PanCon (PanelConnect - piny pro připojení tlačítek a displaye na ovládacím panelu)
\end{itemize} 

Samotná základní deska má dvě verze. Jejich rozdíly jsou vysvětleny níže.
Obě verze desky jsou kromě sekce Control osazeny stejným hardwarem, tedy:

\begin{itemize}
    \item 2x H-můstek VNH2SP30
    \item regulátory napětí 7805CV-DG od STMicroelectronics
    \item pinheady pro připojení senzorů, ovládacího panelu a dalších periferií
    \item svorkovnicemi pro připojení napájecích kabelů a silových výstupů
    \item 
\end{itemize}

\paragraph{PPMB32-E}
Vzhledem k~tomu, že je ProtoPlant veřejně dostupný, nebyl jsem si jist, zda by kompletní osazení takto velké desky zvládl i laik. 
Napadlo mě proto vytvořit i druhou desku, na které by byly osazeny dutinkové lišty pro vsazení vývojové ESP32 DevKitC. 
Odpadla by tedy nutnost kompletně osazovat sekci Control. 
Tuto verzi jsem nazval PPMB32-E (označení E od anglického slova Easy - jednoduchý).

\paragraph{PPMB32-F}
Kompletní, samostatná deska. 
Je přímo osazena procesorem ESP32-WROOM32D i programátorem. 
Vzhledem k~nepoužití DevKitu C je má deska nižší profil, tudíž je možné ji umístit i do nižších prostor. 
Integrovaný programátor lze s~pomocí jumperů odpojit a přes programovací piny připojit externí.

\paragraph{Sekce Control}
Jak již bylo zmíněno, tato část desky zahrnuje modul procesoru ESP32-WROOM32D a programovací obvod. 
Ten se skládá z~převodníku USB-UART CP2102N, tranzistorů SS8050-G (sloužících pro reset procesoru), indikačních LED diod a mikro USB konektoru. 
Nachází se zde i jumper pro přepínání mezi externím programátorem a programátorem přímo na desce.

\paragraph{Sekce H-power}
V této části desky se nacházejí H-můstky VNH2SP30 společně s regulátory napětí 7805CV-DG (výstup 5VDC) a LM3940IT-3.3 (výstup 3,3VDC). 
Na verzi PPMB32-F je dále osazen AMS1117-3.3 pro napájení procesoru. 

V dolní části desky se poté nacházejí dva integrované obvody VNH2SP30, z nichž jedno (VNH1) je určeno pro ovládání aktuátorů manipulujících s okny.
VNH2 má několik režimů funkce, podle připojeného výstupu:
\begin{itemize}
    \item disabled (výstupy jsou deaktivovány)
    \item pump (VNH je použito pro spínání čerpadla, případně stykače řídícího čerpadlo)
    \item heating (VNH je použito pro řízení topné spirály)
\end{itemize}

Napájení desky je rozděleno do tří okruhů. První z kterého je napájena řídící elektronika je dále rozdělen na dva subokruhy s odlišným napětím (3,3~VDC a 5~VDC).
Rozsah vstupního napětí pro tento okruh je 7,5~VDC až 18~VDC.
Zbývající 2 okruhy jsou použity pro oddělené napájení jednotlivých výstupů. 
Jejich napájecí rozsahy jsou rozepsány v tabulce níže.
\begin{table}[h]
    \centering
    \begin{tabular}{llll}
        \hline
        \multicolumn{1}{|l|}{\textbf{Parametr}}           & \multicolumn{1}{l|}{\textbf{Min.}} & \multicolumn{1}{l|}{\textbf{Max.}} & \multicolumn{1}{l|}{\textbf{Jednotka}} \\ \hline
        \multicolumn{1}{|l|}{Vstupní napětí}              & \multicolumn{1}{l|}{5,5}           & \multicolumn{1}{l|}{16}            & \multicolumn{1}{l|}{V}                 \\ \hline
        \multicolumn{1}{|l|}{Výstupní napětí}             & \multicolumn{1}{c|}{-}             & \multicolumn{1}{l|}{16}            & \multicolumn{1}{l|}{V}                 \\ \hline
        \multicolumn{1}{|l|}{Výstupní proud}              & \multicolumn{1}{c|}{-}             & \multicolumn{1}{l|}{30}            & \multicolumn{1}{l|}{A}                 \\ \hline
        \multicolumn{1}{|l|}{Maximální kontinuální proud} & \multicolumn{1}{c|}{-}             & \multicolumn{1}{l|}{14}            & \multicolumn{1}{l|}{A}                 \\ \hline
    \end{tabular}
    \caption{Tabulka napájecích rozsahů napájecích větví VNH1 a VNH2}
\end{table}

\subsection{PPSB - Deska se senzory teploty a vlhkosti}
Deska osazená senzory pro měření vzdušné teploty (DS18B20) a vlhkosti (BME280). 

\paragraph{DS18B20}


\section{Hardwarové verze ProtoPlantu a jejich odlišnosti}

\section{Krabice pro řídící elektroniku a jejich interiér}

\subsection{Instalace elektroniky do krabic - tzv. StoryMount}

\subsection{Těsnění}

\subsection{Ochrana elektroniky před vlhkostí}

\subsection{Ochrana před přehřátím}

\newpage