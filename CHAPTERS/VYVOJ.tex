\chapter{Průběh vývoje ProtoPlantu}
Za dobu vývoje ProtoPlant prošel můj systém spoustou velkých změn.
Vývoj byl započat začátkem roku 2018.
Za tu dobu vyšlo již několik verzí softwaru i hardwaru.

\section{ProtoPlant x1.0 až x4.0 (tzv. legacy verze)}
Původní verze ProtoPlantu byly celkem 4 stabilní a několik vývojových.
Tyto 4 stabilní verze jsou ve zkratce popsány níže.

\subsection{Verze x1.0}
Nejstarší funkční verze ProtoPlantu, založená na Arduinu DUE. 
Tato verze byla pouze shluk kabelů, přes které byly k Arduinu připojeny jednotlivé moduly senzorů.
Byla schopna pouze měřit teplotu a s pomocí relátek ovládat aktuátory, které následně otevíraly, nebo zavíraly okna.
Inverze polarity napájení byla řešena použitím třech relé.
Dvou, která křížově spínala kladný pól napájení aktuátorů a třetího, který připojoval zemnící vodič.
Toto zapojení nebylo zcela ideální, vzhledem k nutnosti použití tří relé a tedy i velkému úbytku napětí.
Použité Arduino bohužel další vývoj nepřežilo.

\subsection{Verze x2.0}
Největší změnou oproti předchozí verzi byl přechod od Arduina na ESP32 devkitC.
Software byl kompletně přepsán a některé funkce byly přesunuty

\paragraph{Po hardwarové stránce}
bylo vytvořeno mnoho prototypů řídící elektroniky.
Nejstarší z nich tvořilo několik modulů s čidly teploty a relátkem připojených drátky k ESP32 devkitu C.
Později jsem začal přecházet na univerzální tištěné spoje, kterých na výrobu prototypů padlo přibližně 20 desek.
Díky těmto prototypům jsem získal zkušenosti s prací s ESP32 a mohl dále pokračovat ve vývoji mého systému.

\paragraph{Software původní verze ProtoPlantu}
byl založen na jednom vlákně postupně volající funkce různých knihoven (ať již mnou vytvořených, či převzatých).
Tento postup se příliš neosvědčil, vzhledem k tomu, že nebylo možné zajistit přesné časování spuštění jednotlivých funkcí.

\section{}


\newpage