\documentclass{template/socthesis}

\usepackage{subcaption}
\usepackage{amsmath}
\usepackage{enumitem}

\addbibresource{text.bib}

\titlecz{Automatický skleník podruhé}
\titleen{Automatic greenhouse second time}
\author{Petr Štourač}
\field{10}
\school{Střední průmyslová škola a Vyšší odborná škola Brno, Sokolská, příspěvková organizace}
\mentor{Mgr. Miroslav Burda}
\mentorstatement{Mgr. Miroslava Burdy}

% Změňte, pokud se liší
%\region{Jihomoravský}
\placefooter{Brno 2020}

\begin{document}

\maketitle

\makecopyrightstatement{V~Brně}

\makethanks{Děkuji svému školiteli Mgr. Miroslavu Burdovi za obětavou pomoc, podnětné připomínky a nekonečnou trpělivost, kterou mi během práce poskytoval.}

\pagestyle{empty}

\section*{Anotace}
Zahradničení je dnes naprosto běžnou zájmovou činností. Mnoho lidí mající takovou zálibu je ovšem velmi časově vytížených. Kromě práce se musí starat mnohdy i o rodinu a na péči o rostliny jim často jednoduše nezbývá čas. 

Tato práce navazuje na moji činnost z minulého ročníku SOČ. Cílem původní práce bylo vytvořit univerzální a dostupný systém pro automatizaci skleníku, který by usnadnil péči o~rostliny časově vytíženým lidem. Tehdy jsem vytvořil systém schopný automaticky řídit ventilaci a závlahu ve skleníku, případně spínat topné těleso. Systém jsem později nazval ProtoPlant.
Systém byl tehdy v rannější fázi vývoje a byl zde velký prostor pro jeho vylepšení v mnoha ohledech. V tomto roce jsem se zaměřil na zdokonalování stávajících funkcí a implementaci nových.

\subsection*{Klíčová slova}
greenhouse automation, ESP32, internet of things, ProtoPlant 

\vspace{20mm}

\section*{Annotation}
The goal of this work is

\subsection*{Keywords}
aut

\newpage
\pagestyle{plain}

\tableofcontents % vysází obsah

%%% Začátek práce
\setcounter{figure}{0}
\setcounter{table}{0}
\newpage

% Uvod prace
\chapter*{Úvod}
\addcontentsline{toc}{chapter}{Úvod}

Zahradničení je dnes naprosto běžnou zájmovou činností. Mnoho lidí majících takovou zálibu je ovšem velmi časově vytížených. Kromě práce se musí starat mnohdy i o~rodinu a na péči o~rostliny jim často jednoduše nezbývá čas. Jedním z~těchto lidí je i můj táta, který mě inspiroval k~vytvoření PROTOPlantu - systému pro snadnou a levnou automatizaci skleníku.

Cílem této práce je vytvořit univerzální a dostupný systém pro automatizaci skleníku, který by usnadnil péči o~rostliny časově vytíženým lidem. 

Systémy pro takovouto automatizaci dnes existují, jsou ovšem určené primárně pro velkozemědělství, nikoli pro člověka, který ve skleníku pěstuje několik druhů zeleniny pro sebe proto, aby ji nemusel kupovat v~obchodě, nebo který vlastní menší skleník s~okrasnými rostlinami. 

Samozřejmě, na internetu existuje spousta návodů, jak si nějaký takový „systém“ vyrobit za pomoci Arduina, nepájivého pole a pár drátků.
Takové řešení se mi ovšem nezdá příliš univerzální a pracující lidé nemají mnohdy čas si takto hrát.
Zároveň pro sestavení něčeho takovového potřebují mít určité znalosti v~elektrotechnice.

Kromě toho jsem chtěl, aby bylo možno systém v~budoucnu připojit k~internetu a sledovat jej tak například z~dovolené, případně měnit nastavení.

Při vytváření práce jsem si dal za cíl, aby byl systém:
\begin{itemize}
    \item kompletně open-source
    \item levný
    \item modulární
    \item snadný na ovládání
    \item univerzální
\end{itemize}

Dalším z~cílů tohoto projektu je úspora energií (elektřina, voda), které lze díky automatizaci dosáhnout.
V~průběhu práce jsem systém nazval \B{PROTOPlant}.

\newpage


\newpage
\chapter*{Závěr}
\addcontentsline{toc}{section}{Závěr}

Závěrečná kapitola obsahuje zhodnocení dosažených výsledků se zvlášť vyznačeným vlastním přínosem studenta.
Povinně se zde objeví i zhodnocení z~pohledu dalšího vývoje projektu, student uvede náměty vycházející ze zkušeností s~řešeným projektem a uvede rovněž návaznosti na právě dokončené projekty.

\newpage
\printbibliography[title=Literatura]
\addcontentsline{toc}{section}{Literatura}

\listoffigures
\addcontentsline{toc}{section}{Seznam obrázků}

\listoftables
\addcontentsline{toc}{section}{Seznam tabulek}

\listoflistedequation
\addcontentsline{toc}{section}{Seznam rovnic}

\end{document}
