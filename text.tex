\documentclass{template/socthesis}

\usepackage{subcaption}
\usepackage{amsmath}
\usepackage{enumitem}

\addbibresource{text.bib}

\titlecz{Automatický skleník podruhé}
\titleen{Automatic greenhouse second time}
\author{Petr Štourač}
\field{10}
\school{Střední průmyslová škola a Vyšší odborná škola Brno, Sokolská, příspěvková organizace}
\mentor{Mgr. Miroslav Burda}
\mentorstatement{Mgr. Miroslava Burdy}

% Změňte, pokud se liší
%\region{Jihomoravský}
\placefooter{Brno 2020}

\begin{document}

\maketitle

\makecopyrightstatement{V~Brně}

\makethanks{Děkuji svému školiteli Mgr. Miroslavu Burdovi za obětavou pomoc, podnětné připomínky a nekonečnou trpělivost, kterou mi během práce poskytoval.}

\pagestyle{empty}

\section*{Anotace}
Zahradničení je dnes naprosto běžnou zájmovou činností. Mnoho lidí mající takovou zálibu je ovšem velmi časově vytížených. Kromě práce se musí starat mnohdy i o rodinu a na péči o rostliny jim často jednoduše nezbývá čas. Jedním z těchto lidí je i můj táta, který mě inspiroval k vytvoření ProtoPlantu - systému pro snadnou a levnou automatizaci skleníku. 

Cílem práce je vytvořit univerzální a dostupný systém pro automatizaci skleníku, který by usnadnil péči o~rostliny časově vytíženým lidem. 

\subsection*{Klíčová slova}
automatizace skleníku, ESP32, internet of things, ProtoPlant 

\vspace{20mm}

\section*{Annotation}
TBD

\subsection*{Keywords}
greenhouse automation, ESP32, internet of things, ProtoPlant

\newpage
\pagestyle{plain}

\tableofcontents % vysází obsah

%%% Začátek práce
\setcounter{figure}{0}
\setcounter{table}{0}
\newpage

% Uvod prace
\chapter*{Úvod}
\addcontentsline{toc}{chapter}{Úvod}

Zahradničení je dnes naprosto běžnou zájmovou činností. Mnoho lidí majících takovou zálibu je ovšem velmi časově vytížených. Kromě práce se musí starat mnohdy i o~rodinu a na péči o~rostliny jim často jednoduše nezbývá čas. Jedním z~těchto lidí je i můj táta, který mě inspiroval k~vytvoření PROTOPlantu - systému pro snadnou a levnou automatizaci skleníku.

Cílem této práce je vytvořit univerzální a dostupný systém pro automatizaci skleníku, který by usnadnil péči o~rostliny časově vytíženým lidem. 

Při vytváření práce jsem si dal za cíl, aby byl systém:
\begin{itemize}
    \item kompletně open-source
    \item levný
    \item modulární
    \item snadný na ovládání
    \item univerzální
\end{itemize}

Dalším z cílů tohoto projektu je úspora energií (elektřina, voda), které lze díky automatizaci dosáhnout.

V průběhu vývoje jsem tento projekt nazval PROTOPlant.
Celý zdrojový kód , hardwarové parametry, schémata, tištěné spoje, novinky z vývoje a další informace týkající se PROTOPlantu jsou k dispozici na webu \textit{www.protoplant.cz}.
\newpage


% Zaver prace
\chapter*{Závěr}
\addcontentsline{toc}{chapter}{Závěr}

Záměrem mojí práce bylo vytvořit univerzální systém pro automatizaci skle\-ní\-ku, který je:
\begin{itemize}
    \item open-source
    \item levný
    \item modulární
    \item snadný na ovládání
    \item univerzální
\end{itemize}

Tento cíl se mi podařilo splnit.
Lidé, kteří mají zájem si systém vytvořit najdou veškerou dokumentaci, schémata a~zdrojový kód na webu \textit{\url{www.protoplant.cz}}.

Díky SOČ jsem se naučil pracovat se softwarem pro návrh PCB Autodesk EAGLE.
Zároveň jsem vylepšil své schopnosti v~programování a~získal spoustu dalších zkušeností v~elektrotechnice a~s~prací na takto komplexních projektech.

Co se týče plánů do budoucna, PROTOPlant budu dále rozvíjet.
Mezi mé plány se řadí dokončení rozpracovaných modulů, vylepšování softwaru a~implementace vzdáleného sledování stavu skleníku přes internet například z~dovolené.

\fxnote[author=PŠ]{\textcolor{mygreen}{Přidat kecy o~prodeji PROTOPlantu a velké řeči}}

\newpage

\printbibliography[title=Literatura]
\addcontentsline{toc}{section}{Literatura}

\listoffigures
\addcontentsline{toc}{section}{Seznam obrázků}

\listoftables
\addcontentsline{toc}{section}{Seznam tabulek}

\listoflistedequation
\addcontentsline{toc}{section}{Seznam rovnic}

\end{document}
