\documentclass{template/socthesis}

\usepackage{subcaption}
\usepackage{amsmath}
\usepackage{enumitem}

\addbibresource{text.bib}

\titlecz{Automatický skleník podruhé}
\titleen{Automatic greenhouse second time}
\author{Petr Štourač}
\field{10}
\school{Střední průmyslová škola a Vyšší odborná škola Brno, Sokolská, příspěvková organizace}
\mentor{Mgr. Miroslav Burda}
\mentorstatement{Mgr. Miroslava Burdy}

% Změňte, pokud se liší
%\region{Jihomoravský}
\placefooter{Brno 2020}

\begin{document}

\maketitle

\makecopyrightstatement{V~Brně}

\makethanks{Děkuji svému školiteli Mgr. Miroslavu Burdovi za obětavou pomoc, podnětné připomínky a nekonečnou trpělivost, kterou mi během práce poskytoval.}

\pagestyle{empty}

\section*{Anotace}
Zahradničení je dnes naprosto běžnou zájmovou činností. Mnoho lidí mající takovou zálibu je ovšem velmi časově vytížených. Kromě práce se musí starat mnohdy i o rodinu a na péči o rostliny jim často jednoduše nezbývá čas. 

Tato práce navazuje na moji činnost z minulého ročníku SOČ. Cílem původní práce bylo vytvořit univerzální a dostupný systém pro automatizaci skleníku, který by usnadnil péči o~rostliny časově vytíženým lidem. Tehdy jsem vytvořil systém schopný automaticky řídit ventilaci a závlahu ve skleníku, případně spínat topné těleso. Systém jsem později nazval ProtoPlant.
Systém byl tehdy v rannější fázi vývoje a byl zde velký prostor pro jeho vylepšení v mnoha ohledech. V tomto roce jsem se zaměřil na zdokonalování stávajících funkcí a implementaci nových.

\subsection*{Klíčová slova}
greenhouse automation, ESP32, internet of things, ProtoPlant 

\vspace{20mm}

\section*{Annotation}
The goal of this work is

\subsection*{Keywords}
aut

\newpage
\pagestyle{plain}

\tableofcontents % vysází obsah

%%% Začátek práce
\setcounter{figure}{0}
\setcounter{table}{0}
\newpage

% Uvod prace
\chapter*{Úvod}
\addcontentsline{toc}{chapter}{Úvod}

%\hyphenation{pou-ži-tím před-po-klad-dem}

\newpage


\newpage
\chapter*{Závěr}
\addcontentsline{toc}{section}{Závěr}

Závěrečná kapitola obsahuje zhodnocení dosažených výsledků se zvlášť vyznačeným vlastním přínosem studenta.
Povinně se zde objeví i zhodnocení z~pohledu dalšího vývoje projektu, student uvede náměty vycházející ze zkušeností s~řešeným projektem a uvede rovněž návaznosti na právě dokončené projekty.

\newpage
\printbibliography[title=Literatura]
\addcontentsline{toc}{section}{Literatura}

\listoffigures
\addcontentsline{toc}{section}{Seznam obrázků}

\listoftables
\addcontentsline{toc}{section}{Seznam tabulek}

\listoflistedequation
\addcontentsline{toc}{section}{Seznam rovnic}

\end{document}
