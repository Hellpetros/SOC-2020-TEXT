\documentclass{template/socthesis}

\usepackage{subcaption}
\usepackage{amsmath}
\usepackage{enumitem}
\usepackage{hyperref}
\usepackage{gensymb} % balíček symbolů
\usepackage{booktabs}

\usepackage[toc,page]{appendix}
\usepackage{color} % balíček pro obarvování textů
\usepackage{xcolor}  % zapne možnost používání barev, mj. pro \definecolor
\definecolor{mygreen}{RGB}{0,150,0} % nastavení barev odkazů 
\usepackage{listings} % balíček pro formátování zdrojových kódů 
\usepackage[author=,status=draft]{fixme} % vkládání poznámek  
% dva módy (status): draft (poznámky se zobrazují v PDF) / final (poznámky se nezobrazují v PDF)
\usepackage{multirow}

\lstset { %
    language=C++,
    backgroundcolor=\color{black!5}, % set backgroundcolor
    basicstyle=\footnotesize,% basic font setting
}

\hyphenation{PROTOPlant}

\addbibresource{text.bib}
\nocite{*}

\titlecz{Automatický skleník}
\titleen{Automatic greenhouse}
\author{Petr Štourač}
\field{7}
\school{Střední průmyslová škola a Vyšší odborná škola Brno, Sokolská, příspěvková organizace}
\mentor{Mgr. Miroslav Burda}
\mentorstatement{Mgr. Miroslava Burdy}

% Změňte, pokud se liší
%\region{Jihomoravský}
\placefooter{Brno 2020}


%\usepackage{hyperref} % balíček pro hypertextové odkazy
% \url{www.odkaz.cz}
% \href{http://www.odkaz.cz}{Text který bude jako odkaz}
%\hyperlink{label}{proklikávací_text} - odkaz na text 
% \hypertarget{label}{cíl_odkazu} - cíl odkazu 
 

\begin{document}

\maketitle

\makecopyrightstatement{V~Brně}

\makethanks{Děkuji svému školiteli Mgr. Miroslavu Burdovi za obětavou pomoc, podnětné připomínky a hlavně nekonečnou trpělivost, kterou mi během práce poskytoval. Dále děkuji Kateřině Jelínkové za kontrolu gramatické správnosti a Mgr. Jaroslavu Smékalovi za korekce anglických textů.}

\pagestyle{empty}

\section*{Anotace}
Zahradničení je dnes naprosto běžnou zájmovou činností. Mnoho lidí mající takovou zálibu je ovšem velmi časově vytížených. Kromě práce se musí starat mnohdy i o~rodinu a na péči o~rostliny jim často jednoduše nezbývá čas. Jedním z~těchto lidí je i můj táta, který mě inspiroval k~vytvoření PROTOPlantu -- systému pro snadnou a levnou automatizaci skleníku. 

Cílem práce je vytvořit univerzální a dostupný systém pro automatizaci skleníku, který by usnadnil péči o~rostliny časově vytíženým lidem. 

\subsection*{Klíčová slova}
automatizace skleníku, ESP32, PROTOPlant, automatizace, open-source hardware, open-source software

\vspace{20mm}

\section*{Annotation}
Gardening is a very common hobby today. However, many people who likes this activity doesn't have enough time for it. Beside work, they have to take care of their families and after this, they don't have any time to take care of plants. My dad is this kind of man. And that inspired me to create PROTOPlant -- system for easy and cheap greenhouse automation.

Goal of this thesis is to create universal and available system for greenhouse automation, that will make it easier for these people to take care of their plants.

Goal of this thesis is to create universal and available system for greenhouse automation, that will make it easier for these people to take care of their plants.

%\fxnote[author=PŠ]{Přeložím během dnešního večera}

\subsection*{Keywords}
greenhouse automation, ESP32, PROTOPlant, automation, open-source hardware, open-source software

\newpage
\pagestyle{plain}

\tableofcontents % vysází obsah

%%% Začátek práce
\setcounter{figure}{0}
\setcounter{table}{0}
\newpage

% Uvod prace
\chapter*{Úvod}
\addcontentsline{toc}{chapter}{Úvod}

Zahradničení je dnes naprosto běžnou zájmovou činností. Mnoho lidí majících takovou zálibu je ovšem velmi časově vytížených. Kromě práce se musí starat mnohdy i o~rodinu a na péči o~rostliny jim často jednoduše nezbývá čas. Jedním z~těchto lidí je i můj táta, který mě inspiroval k~vytvoření PROTOPlantu - systému pro snadnou a levnou automatizaci skleníku.

Cílem této práce je vytvořit univerzální a dostupný systém pro automatizaci skleníku, který by usnadnil péči o~rostliny časově vytíženým lidem. 

Při vytváření práce jsem si dal za cíl, aby byl systém:
\begin{itemize}
    \item kompletně open-source
    \item levný
    \item modulární
    \item snadný na ovládání
    \item univerzální
\end{itemize}

Dalším z cílů tohoto projektu je úspora energií (elektřina, voda), které lze díky automatizaci dosáhnout.

V průběhu vývoje jsem tento projekt nazval PROTOPlant.
Celý zdrojový kód , hardwarové parametry, schémata, tištěné spoje, novinky z vývoje a další informace týkající se PROTOPlantu jsou k dispozici na webu \textit{www.protoplant.cz}.
\newpage


% Motivace
\chapter{Motivace}
K~nápadu vytvořit  PROTOPlant mě dovedl můj otec, který má zálibu v~pěs\-to\-vá\-ní orchidejí.
Má na ně i velký skleník, na který ovšem z~pracovních důvodů nemá příliš času.
Napadlo mne tedy, že je jistě mnoho dalších lidí, kteří jsou na tom s~časem velmi podobně, jako můj otec.

Nejdříve jsem přemýšlel pouze nad regulací teploty otevíráním oken.
Později přibylo i zavlažování a nakonec jsem se rozhodl, že z~tohoto nápadu udělám komplexní systém i s~připojením k~internetu.

V~minulém roce jsem na systému začal pracovat intenzivně a účastnil se s~ním i loňského ročníku SOČ.
To mě motivovalo na něm dále pracovat a rozvíjet jej.

\newpage

% Kapitola zabyvajici se konkurenci
\chapter{Konkurence}
To be done.

\newpage

% Popis funkci, aneb „Co to vsechno umi?"
\chapter{Funkce PROTOPlantu, aneb „Co to všechno umí?“}
To be done.

\newpage

% Jednotlive moduly PROTOPlantu
\chapter{Jednotlivé moduly PROTOPlantu}
PROTOPlant je modulární systém -- není tedy jedním velkým celkem se všemi funkcemi přímo zaintegrovanými.
V~této kapitole se zaměřím na podrobný popis jednotlivých modulů.

\section{Řídící jednotka (PPCU)}
PPCU, neboli řídící jednotka je hlavním modulem celého systému.
Samotné PPCU tvoří elektroinstalační box s~krytím IP65.
Na přední části se nachází ovládací panel s~LCD displejem a ovládacími tlačítky.
Z~bočních stran jsou instalovány vodotěsné průchodky pro provlečení kabelů.

Uvnitř se nachází základní deska (viz \autoref{subsec:motherBoard}) a zdroj napájení.

\section{Přídavné moduly PROTOPlantu}
Kromě samotné řídící elektroniky je možno PROTOPlant rozšířit i o~přídavné moduly. 
Na vývoji těchto modulů se zatím stále pracuje.
Těchto modulů existuje hned několik:

\begin{itemize}
    \item Komunikační a napájecí modul (CIPM -- Communication Interface and Power Module -- modul potřebný pro drátové připojení ostatních modulů, viz \autoref{subsec:CIPM})
    \item Napájecí rozdělovač (PSpl -- Power Splitter)
    \item Modul určený pro měření půdní vlhkosti (SHSM -- Soil Humidity Sensorics Module -- modul vybavený senzory pro měření vlhkosti půdy, viz \autoref{subsec:SHSM})
    \item Modul rozšíření pokrytí senzorikou (SEM -- Sensorics Expansion Module -- modul pro zvýšení počtu senzorů připojených k~PROTOPlantu, viz \autoref{subsec:SEM})
    \item Modul ovládání čerpadel (PCM -- Pump Control Module -- modul pro sledování hladiny vody v~nádrži a ovládání čerpadla, viz \autoref{subsec:PCM})
    \item Modul vzdáleného ovládání (RCM -- Remote Control Module -- modul pro připojení vzdáleného ovládacího panelu, viz \autoref{subsec:RCM})
\end{itemize}
Zjednodušené schéma zapojení a funkce jednotlivých modulů naleznete na \autoref{fig:add-MODULES}.

\paragraph{Napájení přídavných modulů}
je prováděno ve čtyřech režimech.
\begin{itemize}
    \item napájení přímo z~řídící jednotky
    \item napájení z~externího zdroje přes CIPM
    \item napájení přes PSpl
    \item napájení každého modulu odděleně
\end{itemize}

\subparagraph{Napájení přimo z~řídící jednotky}
je možno použít pouze tehdy, když je připojen maximálně jeden modul a to z~důvodu, aby bylo zabráněno podpětí celého systému.
Modul je takto připojen přímo k~napájecímu okruhu A~řídící jednotky (viz \autoref{par:PowerCircuitA}).

\subparagraph{Použití externího zdroje připojeného k~CIPM}
\label{subpar:suplyingViaCIPM}
je použitelné v~případě, kdy uživatel upřednostňuje kabelovou komunikaci mezi moduly a řídící jednotkou.
Při napájení v~tomto režimu je počet připojitelných modulů omezen pouze výkonem zdroje napájení připojeného k~CIPM.
Na vstupní napájecí svorkovnici je připojen externí zdroj. 
Kromě dvou kabelů pro komunikaci jsou na výstupu připojeny i napájecí kabely od jednotlivých modulů.

\subparagraph{Napájení přes PSpl}
funguje na velmi podobném principu, jako předchozí varianta. 
K~PSpl je na vstupní svorkovnici připojen externí zdroj napájení.
Na výstupních svorkovnicích jsou připojeny napájecí kabely jednotlivých modulů.
Tato metoda je určena primárně pro bezdrátovou komunikaci mezi řídící jednotkou a moduly.

\subparagraph{Napájení každého modulu odděleně}
je nejjednodušší metoda napájení.
Každý z~modulů je připojen vlastním kabelem přímo k~napájecímu zdroji.
Primárně je určena pro bezdrátovou komunikaci mezi jednotlivými moduly. 

\subsection{Komunikační a napájecí modul -- CIPM}
\label{subsec:CIPM}
Tento modul funguje jako propojovací uzel mezi řídící jednotkou a všemi přídavnými moduly.
Dále slouží pro připojení externího napájení pro jednotlivé další moduly (viz \autoref{subpar:suplyingViaCIPM}).

\subsection{Modul měření vlhkosti půdy -- SHSM}
\label{subsec:SHSM}
Modul určený pro připojení senzorů měřících vlhkost půdy.
Tento modul je zatím stále ve stádiu konceptu.

\subsection{Modul rozšíření senzoriky -- SEM}
\label{subsec:SEM}
Modul určený pro zvýšení pokrytí prostoru skleníku přidáním dalších enviromentálních senzorů.
Pro malé, případně středně velké skleníky není tento modul potřebný. 
Ve velkých sklenících již své uplatnění najde vzhledem k~tomu, že je vybaven vlastní řídící elektronikou a jediným omezením je dosah zvoleného způsobu komunikace s~řídící jednotkou.

\subsection{Modul řízení čerpadel -- PCM}
\label{subsec:PCM}
Valná většina zahrádkářů má pro svůj skleník i nádrž na vodu.
Tento modul je určen pro sledování hladiny vody v~ní a případné spínání čerpadla, které má za úkol v~nádrži vodu doplňovat.

\subsection{Modul vzdáleného ovládání -- RCM}
\label{subsec:RCM}
Byl vytvořen pro zjednodušení nastavení a ovládání PROTOPlantu.
Skládá se ze dvou částí. 
Komunikační části, kterou lze připojit k~základní desce PROTOPlantu a ovládacího panelu. 
Uživatel ovládací panel nainstaluje na zeď přímo v~domě a může díky němu vzdáleně ovládat celýPROTOPlant přímo z~pohodlí domova. 

\subsection{Uložení řídící elektroniky}
Řídící elektronika (základní deska, řadiče, kabeláž atp.) je uložena v~průmyslových elektroinstalačních boxech s~krytím IP65 (\It{úplná prachotěsnost a odolnost proti tryskající vodě} \cite{IP_ratings}).
Vyvedení kabelů z~těchto boxů je řešeno s~pomocí kabelových průchodek se stejnou úrovní krytí.

Upevnění řídící elektroniky do těchto boxů je řešena díly vytisknutými na 3D tiskárně z~materiálu PET-G.
Ten jsem zvolil pro jeho odolnost a nehygrofilnost.

Konstrukci pro upevnění tvoří zpravidla 2 části:
\begin{itemize}
    \item montážní deska
    \item kabelový unašeč
\end{itemize}

\noindent\B{Montážní deska} je největší částí celého držáku. 
Na spodní straně se nachází drážky pro správné umístění do boxu.
Z~jedné z~bočních stran se nachází drážky pro umístění kartuše se silikagelem.
Ze strany směřující do volného prostoru boxu se nacházejí výstupky, které se zasunou do drážek v~montážní desce pro zdroj.
Na horní straně jsou umístěny otvory pro připevnění základní desky a drážky pro upevnění kabelových unašečů.

\noindent\B{Kabelové unašeče} jsou částí složenou z~více menších dílů.
Jejich úkolem je upevnění kabelů do větších svazků pro vyšší přehlednost.

%\section{Ochrana elektroniky před přehřátím a vlhkostí}
%To be done.

\newpage

% Prubeh vyvoje ProtoPlantu
\chapter{Průběh vývoje ProtoPlantu}



\newpage

% Testovani protoplantu
\chapter{Dlouhodobé testování}
Již přibližně rok je ProtoPlant spuštěn v našem domácím skleníku s orchidejemi.
Během této doby jsem na této jednotce testoval stabilitu a funkčnost celého systému. 
Zároveň jsem zde pravidelně aktualizoval software a s pomocí této testovací jednotky jsem zjišťoval chyby v softwaru a mohl je průběžně opravovat.

\section{Testování napájení solární energií}
Jednotka ProtoPlantu v tomto skleníku je napájena čistě solární energií získávanou ze dvou solárních panelů instalovaných na střeše přilehlého zahradního domku.
Přebytečná energie je ukládána do dvou olověných akumulátorů s napětím 12~VDC.
Z nich je systém napájen v noci, nebo v počasí se špatnou viditelností.

Díky tomu je tato jednotka \B{naprosto nezávislá} na dodávce energie z elektrické sítě.
Jednotka tedy běží i v případě výpadku energie a její náklady na provoz jsou tedy \B{nulové}.

Zásobování vodou řeší sběr dešťové vody ze střechy nedalekého rodinného domu do nádrže instalované pod zemí vedle skleníku samotného.

\section{Průběh testů}
Za dobu fungování proběhlo několik testů v kratších i delších obdobích. \newline

\noindent\B{Co se týče kratších testů}, jejich délka byla v řádech dní, či týdnů. 
Tyto testy probíhaly většinou s cílem testování stability softwaru.
V jejich průběhu mnohdy probíhaly různé opravy softwaru, případně nápravy případných hardwarových nedostatků. \newline

\noindent\B{Dlouhodobé testy} probíhaly většinou s cílem otestovat přesnost senzorů, stabilitu systému, či jeho energetickou náročnost. 

\newpage

% Uspory dosazene pouzitim protoplantu
\chapter{Úspora energií dosažená s~pomocí PROTOPlantu}
Díky možnosti napájet PROTOPlant \B{kompletně s~pomocí solární energie} lze ušetřit elektrickou energii. 
V~noci jej lze napájet z~baterií, do kterých se přebytková energie ze solárních panelů ukládá.
Tento způsob napájení již dlouhou dobu testuji na našem domácím skleníku. 
O~výsledcích se dočtete v~kapitole \ref{sec:SolarPower}.

O solárních panelech se obecně ví, že jejich počáteční náklady jsou poměrně vysoké, ovšem existují způsoby, jak ušetřit i v tomto ohledu.
Existuje mnoho lidí, kteří prodávají solární panely z druhé ruky například ze zrušených solárních farem.
Tyto panely sice mají nižší účinnost, ovšem je otestované, že PROTOPlant dokáží napájet bez problému.
Společně s těmito panely se dají zakoupit i použité akumulátory (povětšinou NiCd FERAK).
Z vlastní zkušenosti mohu potvrdit, že tyto akumulátory jsou ideální pro skladování energie získané pomocí solárních panelů.

Dále lze díky němu šetřit vodou, protože se vždy zavlažuje pouze ve chvíli, kdy je to potřeba.

\newpage

% Dostupnost a distribuce
\chapter{Dostupnost a distribuce PROTOPlantu}
Jak popisuji v~úvodu, jedním z~cílů, který jsem si dal na začátku práce na PROTOPlantu bylo šíření pod licencí open-source.
Toto jsem dodržel. 
Celý software PROTOPlantu je šířen pod licencí MIT, ostatní části (Hardware, atd.) včetně textu této práce je poté pod CC BY-NC-SA 4.0.
Co se týče hardwaru, převzatý hardware (senzory, procesor a další moduly) je z~licence vyňat.
Člověk, který tedy elektrotechnice a programování rozumí si poté může PROTOPlant bez problému sestavit sám v~pohodlí domova.

Ovšem stále je obrovské množství lidí, kteří na sestavení PROTOPlantu nemusí mít dostatečné znalosti, nebo nemají čas si jej sestavovat.
Z~toho důvodu plánuji zahájit výrobu a distribuci mého systému jakožto hotových komponent, které stačí nainstalovat a zapojit.
Ke dni \It{20. 2. 2020} jsem ve fázi, kdy připravuji výrobní podklady jednotlivých komponent, a provádím kroky vedoucí k~založení podniku.
Již v~minulém roce jsem provedl průzkum, který ukázal, že o~PROTOPlant by skutečně byl zájem.

\section{Případové studie}
Pro názorný příklad použití PROTOPlantu uvádím konkrétní případové studie.

\subsection{Malý skleník s užitkovými rostlinami}
Zahrádkář pěstující plodiny čistě pro zásobování sebe a své rodiny čerstvou zeleninou vlastní skleník s plochou obdělávané půdy 6x3 metry.
Výška skleníku jsou přibližně dva metry, jeho objem je tedy vzhledem k tvaru skleníku \B{menší, než 36~m\superscript{3}}.
Pro dostatečné pokrytí prostoru senzorikou tedy stačí dva senzory teploty (viz kapitola \ref{sec:DS18B20}) a dva senzory vzdušné vlhkosti (viz kapitola \ref{sec:DHT22}). 
Vzhledem k obdělávané ploše, stačí použít 4 senzory půdní vlhkosti.
Zavlažování je vzhledem k pěstovaným rostlinám řešeno děrovanou hadicí položenou přímo na půdě.

\fxnote[author=PŠ]{Zde přibudou fotografie babiččina skleníku a tabulka s cenami (dnes večer to spočítám)}

\subsection{Středně velký skleník s okrasnými rostlinami}
Skleník, ve kterém PROTOPlant průběžně testuji je určen pro pěstování orchidejí. 
Jeho plocha je přibližně 10~x~4~m.
Většina rostlin je v květináčích s molitanem či substrátem zavěšena v prostoru, nebo položena na stole.
Teplotní senzory jsou zde použity 4, senzory vlhkosti 3.
Vzhledem k tomu, že většina z těchto rostlin je epifytní, nesnášejí trvale vlhkou půdu. 
Z tohoto důvodu je zavlažování řešeno rozprašovačem umístěným pod stropem skleníku a upraveným nastavením jeho spínání.
Neprobíhá tedy tak často.

\fxnote[author=PŠ]{Zde přibude tabulka s cenou}

\begin{figure}[htbp]
    \centering
    \includegraphics[width=\textwidth]{img/PHOTOS/OrchidHouse_EXTERIOR.jpg}
    \caption{Exteriér testovacího skleníku}
    \label{fig:OrchidHouse_EXTERIOR}
\end{figure}

\begin{figure}[htbp]
    \centering
    \includegraphics[width=\textwidth]{img/PHOTOS/OrchidHouse_INTERIOR.jpg}
    \caption{Interiér testovacího skleníku}
    \label{fig:OrchidHouse_INTERIORs}
\end{figure}
\newpage

% Zaver prace
\chapter*{Závěr}
\addcontentsline{toc}{chapter}{Závěr}

Záměrem mojí práce bylo vytvořit univerzální systém pro automatizaci skle\-ní\-ku, který je:
\begin{itemize}
    \item open-source
    \item levný
    \item modulární
    \item snadný na ovládání
    \item univerzální
\end{itemize}

Tento cíl se mi podařilo splnit.
Lidé, kteří mají zájem si systém vytvořit najdou veškerou dokumentaci, schémata a~zdrojový kód na webu \textit{\url{www.protoplant.cz}}.

Díky SOČ jsem se naučil pracovat se softwarem pro návrh PCB Autodesk EAGLE.
Zároveň jsem vylepšil své schopnosti v~programování a~získal spoustu dalších zkušeností v~elektrotechnice a~s~prací na takto komplexních projektech.

Co se týče plánů do budoucna, PROTOPlant budu dále rozvíjet.
Mezi mé plány se řadí dokončení rozpracovaných modulů, vylepšování softwaru a~implementace vzdáleného sledování stavu skleníku přes internet například z~dovolené.

\fxnote[author=PŠ]{\textcolor{mygreen}{Přidat kecy o~prodeji PROTOPlantu a velké řeči}}

\newpage
\newpage

\appendix
\addcontentsline{toc}{chapter}{Přílohy}
% Tistene spoje a elektronika
\input{CHAPTERS/ELECTRONICS_and_PCBs.tex}

% Podrobny popis softwaru
\chapter*{Úvod}
\addcontentsline{toc}{chapter}{Software}

\newpage

% Prilohy
\chapter*{Přílohy}
\addcontentsline{toc}{chapter}{Přílohy}

% chapter HARDWARE, modules
\begin{figure}[htbp]
    \centering
    \includegraphics[angle=90,origin=c,scale=0.7]{img/HARDWARE/MODULES.png}
    \caption{Schéma zapojení a funkce jednotlivých modulů}
    \label{fig:add-MODULES}
 \end{figure}

 \begin{figure}[h]
    \centering
    \includegraphics[width=\textwidth]{img/HARDWARE/PPSB-T_BOTH.png}
    \caption{Vizualizace PPSB-T (horní strana vpravo, dolní vlevo)}
    \label{fig:PPSB-T_VISUAL}
\end{figure}

\begin{figure}[h]
    \centering
    \includegraphics[width=\textwidth]{img/HARDWARE/PPSB-TH_BOTH.png}
    \caption{Vizualizace desky PPSB-TH (horní strana vlevo, dolní vpravo)}
    \label{fig:PPSB-TH_VISUAL}
\end{figure}

\newpage

\printbibliography[title=Literatura]
\addcontentsline{toc}{chapter}{Literatura}

\listoffigures
\addcontentsline{toc}{section}{Seznam obrázků}

\listoftables
\addcontentsline{toc}{section}{Seznam tabulek}

\end{document}
